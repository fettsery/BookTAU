\chapter{Синтез линейных непрерывных систем}
\setcounter{section}{8} %TODO:Remove
\setcounter{subsection}{2} %TODO:Remove
\setcounter{equation}{23} %TODO:Remove
\subsection{Идентификационное каноническое представление системы с одним (скалярным) выходом}
С помощью рассуждений, аналогичных проведённым в п.3.8.1, можно получить следующие результаты.\\
Если пара матриц    полностью наблюдаема,  то  в   пространстве состояний  Х  всегда существует базис, в котором пара ${A,C}$  имеет идентификационное каноническое представление (ИКП):
\begin{equation}
	A_I = 
	\begin{bmatrix}
	    0 & 0 & \dots & 0 & 0 & -\alpha_n \\
	    1 & 0 & \dots & 0 & 0 & -\alpha_{n-1} \\
	    0 & 1 & \dots & 0 & 0 & -\alpha_{n-2} \\
	    \dots & \dots & \dots & \dots & \dots & \dots \\
	    0 & 0 & \dots & 0 & 0 & -\alpha_3 \\
	    0 & 0 & \dots & 1 & 0 & -\alpha_2 \\
	    0 & 0 & \dots & 0 & 1 & -\alpha_1
	\end{bmatrix}
\end{equation}

\begin{equation}
	C_I = 
	\begin{bmatrix}
	    0 & 0 & \dots & 0 & 0 & 1
	\end{bmatrix}
\end{equation}

Отметим, что

\begin{equation}
	A_I^T = A_U; C_I^T=\vec{b}_U.
\end{equation}

Если   в  некотором  исходном  базисе [$h$] заданы матрицы $A_H, C_H$ и если система полностью наблюдаема, то, для того чтобы вычислить их (матриц) ИКП, достаточно вычислить коэффициенты характеристичес¬кого полинома $\phi_A(\lambda)$ . После этого может быть вычислена матрица преобразования от исходного базиса [$h$] к ИКП в соответствии с (3.7.13):

\begin{equation}
	I_H^{-1}=N_I^{-1}N_H.
\end{equation}

Если известна матрица $B_H$ при векторе управления в исходном базисе, то с учётом (3.6.8)  в базисе ИКП она может быть определена с помощью соотношения

\begin{equation}
	B_I=I_H^{-1}B_H
\end{equation}

\subsection{Передаточная функция и структура для системы в ИКП}